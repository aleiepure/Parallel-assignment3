\section{\label{sec:transposition}Parallel matrix transposition}

This exercise involves the implementation of two matrix transposition algorithms %
using MPI (Message Passing Protocol) for parallelization. The goal is to compare %
the scalability and efficiency between both versions.

Transposing a matrix involves flipping it over its main diagonal, exchanging %
rows and columns, resulting in a new matrix where the rows of the original one %
become the columns of the transposed matrix and vice versa.\\
When handling sequential codes, this is relatively straightforward. However, %
the introduction of MPI parallelization complicates the implementation.

For part 1, the initial implementation mirrored that of the previous assignment. %
Improving it with MPI for part 2 proved more challenging. It required the %
evaluation of various strategies to achieve a correct data subdivision. %
What follows is a concise breakdown of the thinking process that led me to the %
implemented logic.

\begin{wrapfigure}{l}{0.4\textwidth}%
    \vspace{-0.8cm}
    \centering%
    \caption{\label{image:pointers-to-pointers}Idea 1 - Array of pointers to arrays}%
    \includegraphics[width=0.4\linewidth]{pointers-to-pointers.tex}%
\end{wrapfigure}%
The first concept was to leverage the way \textit{C} stores matrices as an array %
of pointers to other arrays (Figure \ref{image:pointers-to-pointers}). However, %
I abandoned this idea quickly due to the challenges in passing such data structure to %
MPI's collective operation functions (like \texttt{MPI\_Scatter()}, %
\texttt{MPI\_Gather()}, etc.). This led me to treat the matrix as a one-dimensional %
array (Figure \ref{image:matrix-as-array}). This turned out to be a winning approach %
and ultimately what I implemented.\\%
\begin{figure}[h]%
    \centering%
    \caption{\label{image:matrix-as-array}Idea 2 - Matrix as a 1D array}%
    \includegraphics[width=0.7\linewidth]{matrix-as-array.tex}%
\end{figure}%

In task 3, I opted for a different approach: I divided the matrix %
(still represented as a 1-D array) into sizable blocks of equal dimensions. Each of these %
blocks is further subdivided and distributed to the available CPUs as smaller blocks. %
These smaller blocks are the local copy of each processor and are transposed and %
sent back to the main CPU. The latter is responsible for reassembling the transposed %
smaller portions into the transposed large block. \\%
Before iterating to the next block, the transposed block is positioned correctly %
to create the final transposed matrix, also stored in a%
1-D array. All matrices are assumed to be square, and the number of blocks dividing the matrix %
is equal to the number of processors. While both limitations are present for simplicity, %
with small modifications any number can be used, provided it is mathematically %
compatible.
\begin{figure}[h]
    \caption{\label{image:transpose-blocks}Idea 3 - Divide into blocks, each processed individually}
    \centering
    \includegraphics[width=0.7\linewidth]{transpose-blocks.tex}
\end{figure}

Although the assignment asked for a single executable with explicit functions for %
the sequential and parallel versions of the 2 algorithms, I could not find a reliable %
way to divide my logic into separate functions. Due to time constraints, I decided %
to write them as 2 separate programs. I interpreted the sequential requirement as %
the execution of the same parallel code restricted to only one CPU. Each program is called %
repeatedly using a series of PBS files - \textit{bash} scripts that detail the 
necessary resources required for the execution on the cluster.

\subsection*{Result analysis}
The final code and supporting files were uploaded to the University's HPC cluster %
and queued for execution. The nodes responsible for the computations were %
\texttt{hpc-c11-node12.unitn.it} (for 1-CPU and 4-CPUs executions), %
\texttt{hpc-c11-node01.unitn.it}, and \texttt{hpc-c04-node01.unitn.it}. The first three %
nodes are equipped with an Intel\textsuperscript{\textregistered} %
Xeon\textsuperscript{\textregistered} Gold 6252N CPU running at 2.30GHz and 1024GB %
of RAM, while the fourth has an Intel\textsuperscript{\textregistered} %
Xeon\textsuperscript{\textregistered} Gold 6140M CPU clocked at 2.30GHz and 768GB %
of RAM.

To meet the assignment's requirement for both strong and weak scaling, two sets of %
input matrix sizes were used. The weak scaling has progressively increasing %
workloads, while the strong scaling maintains them constant. Table \ref{table:scale-weak} %
and table \ref{table:scale-strong} report the execution times for both weak %
and strong scaling respectively, alongside the achieved bandwidth, %
calculated using Equation \ref{eq:bandwidth}. Additionally, plot \ref{plot:weak-scale} %
and plot \ref{plot:strong-scale} visualize these results.

\begin{equation}
    \label{eq:bandwidth}
    \text{Bandwidth} = \frac{B_r * B_w * \text{Times} * \text{size\_of(\textit{float})}}{\text{Run Time}} = 
    \begin{cases}
        \frac{\text{Size}^2 * 3 * 4}{\text{Run Time}}, & \text{without blocks}\\
        \frac{\text{Size}^2 * 4 * 4}{\text{Run Time}}, & \text{with blocks}
    \end{cases}\qquad\text{[Byte/s]}
\end{equation}

Unfortunately, I had difficulties obtaining results for the missing values in the %
tables above as the execution would stall until the allocated wall time was reached, %
resulting in the process being terminated by the system. This is probably happening %
because the assigned nodes have insufficient memory available.

\begin{figure}[h!tb]%
    \centering
    \captionsetup{type=plot}
    \caption{\label{plot:weak-scale}Weak scale graph by size - CPUs ratio}
    \begin{tikzpicture}%
        \begin{groupplot}[
                group style={
                    group size=2 by 1,
                    group name=plot,
                    xlabels at=edge bottom,
                    ylabels at=edge left,
                    vertical sep=1.5cm,
                    horizontal sep=1.5cm,
                }, 
                width={0.45\linewidth},
                xlabel={CPUs},
                ylabel={Time [s]},
                xtick={1, 4, 16, 64},
                legend to name=legend,
                legend columns=-1,
                cycle multiindex* list={
                    color list
                        \nextlist
                    marks
                        \nextlist
                },
            ]
            \nextgroupplot[
                title={Normal},
                xmode=log, 
                log ticks with fixed point,
            ]

            \addplot
                coordinates {
                    (1, 0.00003)
                    (4, 0.01306)
                    (16, 0.01504)
                    (64, 0.22652)
                };

            \addplot
                coordinates {
                    (1, 0.00013)
                    (4, 0.03965)
                    (16, 0.02647)
                    (64, 0.72118)
                };
        
            \addplot
                coordinates {
                    (1, 0.00051)
                    (4, 0.01104)
                    (16, 0.13280)
                    (64, 5.16715)
                };

            \addplot
                coordinates {
                    (1, 0.00298)
                    (4, 0.06305)
                    (16, 0.31432)
                };

            \nextgroupplot[
                title={Blocks},
                xmode=log, 
                log ticks with fixed point,
            ]
            \addplot
                coordinates {
                    (1, 0.00012)
                    (4, 0.01700)
                    (16, 0.01117)
                    (64, 0.19813)
                };

            \addplot
                coordinates {
                    (1, 0.00035)
                    (4, 0.18536)
                    (16, 0.04131)
                    (64, 0.90659)
                };
        
            \addplot
                coordinates {
                    (1, 0.00137)
                    (4, 0.09559)
                    (16, 0.15336)
                    (64, 3.69552)
                };

            \addplot
                coordinates {
                    (1, 0.00577)
                    (4, 0.09333)
                    (16, 0.65696)
                    (64, 14.97982)
                };

                \legend{$\frac{64}{1}$, $\frac{128}{1}$, $\frac{256}{1}$, $\frac{512}{1}$}
        \end{groupplot}
    \end{tikzpicture}
    \ref{legend}
\end{figure}
\begin{figure}[h!tb]%
    \centering
    \captionsetup{type=plot}
    \caption{\label{plot:strong-scale}Strong scale graph by size of matrices}
    \begin{tikzpicture}%
        \begin{groupplot}[
                group style={
                    group size=2 by 1,
                    group name=plot,
                    xlabels at=edge bottom,
                    ylabels at=edge left,
                    vertical sep=1.5cm,
                    horizontal sep=1.5cm,
                }, 
                width={0.45\linewidth},
                xlabel={CPUs},
                ylabel={Time [s]},
                xtick={1, 4, 16, 64},
                legend to name=legend,
                legend columns=-1,
                cycle multiindex* list={
                    color list
                        \nextlist
                    marks
                        \nextlist
                },
            ]
            \nextgroupplot[
                title={Normal},
                xmode=log, 
                log ticks with fixed point,
            ]

                \addplot
                    coordinates {
                        (1, 0.00230)
                        (4, 0.00199)
                        (16, 0.00552)
                        (64, 0.01288)
                    };
        
                \addplot
                    coordinates {
                        (1, 0.06648)
                        (4, 0.02901)
                        (16, 0.02207)
                        (64, 0.10043)
                    };
                
                \addplot
                    coordinates {
                        (1, 1.76583)
                        (4, 0.80396)
                        (16, 1.55372)
                        (64, 2.19584)
                    };

            \nextgroupplot[
                title={Blocks},
                xmode=log, 
                log ticks with fixed point,
            ]
                \addplot
                    coordinates {
                        (1, 0.00518)
                        (4, 0.00394)
                        (16, 0.00362)
                        (64, 0.00858)
                    };
                
                \addplot
                    coordinates {
                        (1, 0.11989)
                        (4, 0.05956)
                        (16, 0.04145)
                        (64, 0.05700)
                    };

                \addplot
                    coordinates {
                        (1, 3.69059)
                        (4, 1.42328)
                        (16, 0.63222)
                        (64, 0.87797)
                    };
                
                \addplot
                    coordinates {
                        (1, 53.89544)
                        (4, 31.03135)
                        (16, 11.29802)
                        (64, 15.24146)
                    }; 

                \legend{512, 2048, 8192, 32768}
        \end{groupplot}
    \end{tikzpicture}
    \ref{legend}
\end{figure}

\begin{figure}[h!tb]%
    \centering
    \captionsetup{type=plot}
    \caption{\label{plot:weak-scale}Weak scale graph by size - CPUs ratio}
    \begin{tikzpicture}%
        \begin{groupplot}[
                group style={
                    group size=2 by 1,
                    group name=plot,
                    xlabels at=edge bottom,
                    ylabels at=edge left,
                    vertical sep=1.5cm,
                    horizontal sep=1.5cm,
                }, 
                width={0.45\linewidth},
                xlabel={CPUs},
                ylabel={Time [s]},
                xtick={1, 4, 16, 64},
                legend to name=legend,
                legend columns=-1,
                cycle multiindex* list={
                    color list
                        \nextlist
                    marks
                        \nextlist
                },
            ]
            \nextgroupplot[
                title={Normal},
                xmode=log, 
                log ticks with fixed point,
            ]

            \addplot
                coordinates {
                    (1, 0.00003)
                    (4, 0.01306)
                    (16, 0.01504)
                    (64, 0.22652)
                };

            \addplot
                coordinates {
                    (1, 0.00013)
                    (4, 0.03965)
                    (16, 0.02647)
                    (64, 0.72118)
                };
        
            \addplot
                coordinates {
                    (1, 0.00051)
                    (4, 0.01104)
                    (16, 0.13280)
                    (64, 5.16715)
                };

            \addplot
                coordinates {
                    (1, 0.00298)
                    (4, 0.06305)
                    (16, 0.31432)
                };

            \nextgroupplot[
                title={Blocks},
                xmode=log, 
                log ticks with fixed point,
            ]
            \addplot
                coordinates {
                    (1, 0.00012)
                    (4, 0.01700)
                    (16, 0.01117)
                    (64, 0.19813)
                };

            \addplot
                coordinates {
                    (1, 0.00035)
                    (4, 0.18536)
                    (16, 0.04131)
                    (64, 0.90659)
                };
        
            \addplot
                coordinates {
                    (1, 0.00137)
                    (4, 0.09559)
                    (16, 0.15336)
                    (64, 3.69552)
                };

            \addplot
                coordinates {
                    (1, 0.00577)
                    (4, 0.09333)
                    (16, 0.65696)
                    (64, 14.97982)
                };

                \legend{$\frac{64}{1}$, $\frac{128}{1}$, $\frac{256}{1}$, $\frac{512}{1}$}
        \end{groupplot}
    \end{tikzpicture}
    \ref{legend}
\end{figure}
\begin{figure}[h!tb]%
    \centering
    \captionsetup{type=plot}
    \caption{\label{plot:strong-scale}Strong scale graph by size of matrices}
    \begin{tikzpicture}%
        \begin{groupplot}[
                group style={
                    group size=2 by 1,
                    group name=plot,
                    xlabels at=edge bottom,
                    ylabels at=edge left,
                    vertical sep=1.5cm,
                    horizontal sep=1.5cm,
                }, 
                width={0.45\linewidth},
                xlabel={CPUs},
                ylabel={Time [s]},
                xtick={1, 4, 16, 64},
                legend to name=legend,
                legend columns=-1,
                cycle multiindex* list={
                    color list
                        \nextlist
                    marks
                        \nextlist
                },
            ]
            \nextgroupplot[
                title={Normal},
                xmode=log, 
                log ticks with fixed point,
            ]

                \addplot
                    coordinates {
                        (1, 0.00230)
                        (4, 0.00199)
                        (16, 0.00552)
                        (64, 0.01288)
                    };
        
                \addplot
                    coordinates {
                        (1, 0.06648)
                        (4, 0.02901)
                        (16, 0.02207)
                        (64, 0.10043)
                    };
                
                \addplot
                    coordinates {
                        (1, 1.76583)
                        (4, 0.80396)
                        (16, 1.55372)
                        (64, 2.19584)
                    };

            \nextgroupplot[
                title={Blocks},
                xmode=log, 
                log ticks with fixed point,
            ]
                \addplot
                    coordinates {
                        (1, 0.00518)
                        (4, 0.00394)
                        (16, 0.00362)
                        (64, 0.00858)
                    };
                
                \addplot
                    coordinates {
                        (1, 0.11989)
                        (4, 0.05956)
                        (16, 0.04145)
                        (64, 0.05700)
                    };

                \addplot
                    coordinates {
                        (1, 3.69059)
                        (4, 1.42328)
                        (16, 0.63222)
                        (64, 0.87797)
                    };
                
                \addplot
                    coordinates {
                        (1, 53.89544)
                        (4, 31.03135)
                        (16, 11.29802)
                        (64, 15.24146)
                    }; 

                \legend{512, 2048, 8192, 32768}
        \end{groupplot}
    \end{tikzpicture}
    \ref{legend}
\end{figure}

Looking at the weak scaling results (plot \ref{plot:weak-scale}), we can observe %
that, although the workload of each CPU remains constant, the time required for %
computation increases as the number of CPUs rises. %
This is because communications between the processes increase as well, %
becoming more frequent and resource-intensive. \\%
On the other hand, the strong scaling (plot \ref{plot:strong-scale}) shows a general decreasing %
trend, pronounced for matrices of size 8192 in the %
normal algorithm and 32768 in the block one. In those cases, the curve starts to %
flatten out with executions executed by 4 and 16 CPUs respectively.

The parallel scalability of both algorithms was evaluated using the bandwidth %
as the performance metric. Plot \ref{plot:performance}, plot \ref{plot:speedup}, and %
plot \ref{plot:efficiency} show us a representation of the performance, speedup, and %
efficiency respectively.

\begin{figure}[h!tb]%
    \centering
    \captionsetup{type=plot}
    \caption{\label{plot:performance}Performance graph by size of matrices}
    \begin{tikzpicture}%
        \begin{groupplot}[
                group style={
                    group size=2 by 1,
                    group name=plot,
                    xlabels at=edge bottom,
                    ylabels at=edge left,
                    vertical sep=1.5cm,
                    horizontal sep=1.5cm,
                }, 
                width={0.45\linewidth},
                xlabel={CPUs},
                ylabel={Bandwidth [GB/s]},
                xtick={1, 4, 16, 64},
                legend to name=legend,
                legend columns=-1,
                cycle multiindex* list={
                    color list
                        \nextlist
                    marks
                        \nextlist
                },
            ]
            \nextgroupplot[
                title={Normal},
                xmode=log, 
                log ticks with fixed point,
            ]

                \addplot
                    coordinates {
                        (1, 1.366)
                        (4, 1.583)
                        (16, 0.570)
                        (64, 0.244)
                    };
        
                \addplot
                    coordinates {
                        (1, 0.757)
                        (4, 1.735)
                        (16, 2.280)
                        (64, 0.501)
                    };

                \addplot
                    coordinates {
                        (1, 0.456)
                        (4, 1.002)
                        (16, 0.518)
                        (64, 0.367)
                    };

            \nextgroupplot[
                title={Blocks},
                xmode=log, 
                log ticks with fixed point,
            ]
                \addplot
                    coordinates {
                        (1, 0.810)
                        (4, 1.066)
                        (16, 1.160)
                        (64, 0.489)
                    };
                
                \addplot
                    coordinates {
                        (1, 0.560)
                        (4, 1.127)
                        (16, 1.619)
                        (64, 1.177)
                    };

                \addplot
                    coordinates {
                        (1, 0.291)
                        (4, 0.754)
                        (16, 1.698)
                        (64, 1.223)
                    };
                
                \addplot
                    coordinates {
                        (1, 0.319)
                        (4, 0.554)
                        (16, 1.521)
                        (64, 1.127)
                    }; 

                    \legend{512, 2048, 8192, 32768}
        \end{groupplot}
    \end{tikzpicture}
    \ref{legend}
\end{figure}
\begin{figure}[h!tb]%
    \centering
    \captionsetup{type=plot}
    \caption{\label{plot:speedup}Speedup graph by size of matrices}
    \begin{tikzpicture}%
        \begin{groupplot}[
                group style={
                    group size=2 by 1,
                    group name=plot,
                    xlabels at=edge bottom,
                    ylabels at=edge left,
                    vertical sep=1.5cm,
                    horizontal sep=1.5cm,
                }, 
                width={0.45\linewidth},
                xlabel={CPUs},
                ylabel={Speedup [$\times$]},
                xtick={1, 4, 16, 64},
                legend to name=legend,
                legend columns=-1,
                cycle multiindex* list={
                    color list
                        \nextlist
                    marks
                        \nextlist
                },
            ]
            \nextgroupplot[
                title={Normal},
                xmode=log, 
                log ticks with fixed point,
            ]

                \addplot
                    coordinates {
                        (1, 1)
                        (4, 1.16)
                        (16, 0.42)
                        (64, 0.18)
                    };
        
                \addplot
                    coordinates {
                        (1, 1)
                        (4, 2.29)
                        (16, 3.01)
                        (64, 0.66)
                    };

                \addplot
                    coordinates {
                        (1, 1)
                        (4, 2.20)
                        (16, 1.14)
                        (64, 0.80)
                    };

            \nextgroupplot[
                title={Blocks},
                xmode=log, 
                log ticks with fixed point,
            ]
                \addplot
                    coordinates {
                        (1, 1)
                        (4, 1.32)
                        (16, 1.43)
                        (64, 0.60)
                    };
                
                \addplot
                    coordinates {
                        (1, 1)
                        (4, 2.01)
                        (16, 2.89)
                        (64, 2.10)
                    };

                \addplot
                    coordinates {
                        (1, 1)
                        (4, 2.59)
                        (16, 5.84)
                        (64, 4.20)
                    };
                
                \addplot
                    coordinates {
                        (1, 1)
                        (4, 1.74)
                        (16, 4.77)
                        (64, 3.54)
                    }; 

                \legend{512, 2048, 8192, 32768}
        \end{groupplot}
    \end{tikzpicture}
    \ref{legend}
\end{figure}
\begin{figure}[h!tb]%
    \centering
    \captionsetup{type=plot}
    \caption{\label{plot:efficiency}Efficiency graph by size of matrices}
    \begin{tikzpicture}%
        \begin{groupplot}[
                group style={
                    group size=2 by 1,
                    group name=plot,
                    xlabels at=edge bottom,
                    ylabels at=edge left,
                    vertical sep=1.5cm,
                    horizontal sep=1.5cm,
                }, 
                width={0.45\linewidth},
                xlabel={CPUs},
                ylabel={Efficiency [\%]},
                xtick={1, 4, 16, 64},
                legend to name=legend,
                legend columns=-1,
                cycle multiindex* list={
                    color list
                        \nextlist
                    marks
                        \nextlist
                },
            ]
            \nextgroupplot[
                title={Normal},
                xmode=log, 
                ymode=log,
                log ticks with fixed point,
            ]

                \addplot
                    coordinates {
                        (1, 100)
                        (4, 28.98)
                        (16, 2.61)
                        (64, 0.28)
                    };
        
                \addplot
                    coordinates {
                        (1, 100)
                        (4, 57.28)
                        (16, 18.82)
                        (64, 1.03)
                    };

                \addplot
                    coordinates {
                        (1, 100)
                        (4, 54.91)
                        (16, 7.10)
                        (64, 1.26)
                    };

            \nextgroupplot[
                title={Blocks},
                xmode=log, 
                ymode=log, 
                log ticks with fixed point,
            ]
                \addplot
                    coordinates {
                        (1, 100)
                        (4, 32.92)
                        (16, 8.96)
                        (64, 0.94)
                    };
                
                \addplot
                    coordinates {
                        (1, 100)
                        (4, 50.32)
                        (16, 18.08)
                        (64, 3.29)
                    };

                \addplot
                    coordinates {
                        (1, 100)
                        (4, 64.83)
                        (16, 36.48)
                        (64, 6.57)
                    };
                
                \addplot
                    coordinates {
                        (1, 100)
                        (4, 43.42)
                        (16, 29.81)
                        (64, 5.53)
                    }; 

                \legend{512, 2048, 8192, 32768}
        \end{groupplot}
    \end{tikzpicture}
    \ref{legend}
\end{figure}

Contrary to what we would expect, the performance of both algorithms is quite %
irregular. Typically, the bandwidth should rise with the increase in CPUs, but %
in this case, it increases up to a point before declining. What is said holds %
valid for both the normal and the blocks algorithms. All the above is also %
reflected in the speedup and efficiency graphs, which show data scattered all %
over the place with a general downward trend with the increase in the number of CPUs.
